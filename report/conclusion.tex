\part{Conclusion et Bilan}
\setcounter{section}{0}

\section{Bilan du chef de projet}

\subsection{Bilan global}
Le projet s'est dans l'emsemble bien déroulé. Le principal problème de ce projet a été de constamment devoir recommencer les différents diagrammes jusqu'à ce qu'ils soient corrects. De plus, il existe peu d'outils qui soient réellement pratiques pour la réalisation de ces projets, surtout en ce qui concerne les IHM.

\subsection{Outils utilisés}
Dans le but d'harmoniser l'ensemble du rendu, nous avons choisi d'utiliser \textbf{PlantUML} pour réaliser l'ensemble des diagrammes. La possibilité d'utiliser des variables pour les différents diagrammes nous a permis de détecter et d'éviter les redondances au niveau des services métiers. Le principal problème avec cet outil est qu'il est difficile de s'éloigner de l'UML standard (exemple : impossible de créer plusieurs "endPoints" dans les diagrammes d'activités).\\

Les IHM ont été réalisées à l'aide de \textbf{Balsamiq}, qui est malheureusement un des seuls outils qui permettent d'aller assez rapidement. Mais l'export pdf final est très décevant au niveau qualité. 

\subsection{Bilan sur le temps de travail}
Le lecteur peut trouver ci-dessous un graphique indiquant le temps mis par l'équipe sur chaque tâche.
On constate un dépassement du temps estimé pour la réalisation du projet, en particulier du au fait qu'il a fallu recommencer plusieurs fois les diagrammes à cause de certains aspects métiers peu clairs et/ou mal compris (ex : contact prévu/affecté, plusieurs rendez-vous par contact, organisation de la proposition commerciale, etc...)


\todo{insérer un graphique des temps}


\section{Conclusion}